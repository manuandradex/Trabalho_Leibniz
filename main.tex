\documentclass{article}

\usepackage{graphicx} % Required for inserting images
\usepackage{listings}
\usepackage{xcolor}

\lstset{
language=Python,
basicstyle=\ttfamily,
keywordstyle=\color{blue},
commentstyle=\color{green},
stringstyle=\color{red},
showstringspaces=false,
breaklines=true,
tabsize=4
}

\begin{document}
\begin{center}

\subsection{TRABALHO DE MATEMÁTICA}

MANUELA SILVA DE ANDRADE - 1º CICLO DE CIÊNCIA DE DADOS

\end{center}


1. Deduza A o determinante 4x4 usando a fórmula:

\begin{figure}[h]
    \centering
    \includegraphics{Captura de tela 2023-05-09 075535.png}
    \label{fig:my_label}
\end{figure}

\begin{lstlisting}    
A = [a1 a2 a3 a4]
    [a5 a6 a7 a8]
    [a9 a10 a11 a12]
    [a13 a14 a15 a16]

det(A) =
a1*a6*a11*a16 + a1*a7*a12*a14 + a1*a8*a10*a15 + a2*a5*a12*a15
+ a2*a7*a9*a16 + a2*a8*a11*a13 + a3*a5*a10*a16 + a3*a6*a12*a13
+ a3*a8*a9*a14 + a4*a5*a11*a14 + a4*a6*a9*a15 + a4*a7*a10*a13
- a1*a6*a12*a15 - a1*a7*a10*a16 - a1*a8*a11*a14 - a2*a5*a11*a16
- a2*a7*a12*a13 - a2*a8*a9*a15 - a3*a5*a12*a14 - a3*a6*a9*a16
- a3*a8*a10*a13 - a4*a5*a10*a15 - a4*a6*a11*a13 - a4*a7*a9*a14

Exemplo:

A = [1 0 3 2]
    [2 1 1 0]
    [0 2 0 1]
    [-1 4 2 1]

det(A) = ( 1*1*0*1 + 1*1*1*4 + 1*0*2*2 + 0*2*1*2 + 0*1*0*1 + 0*0*0*(-1) + 3*2*2*1 + 3*1*1*(-1) + 3*0*0*4 + 2*2*0*4 + 2*1*0*2 + 2*1*2*(-1) - 1*1*1*2 - 1*1*2*1 - 1*0*0*4 - 0*2*0*1 - 0*1*1*(-1) - 0*0*0*2 - 3*2*1*4 - 3*1*0*1 - 3*0*2*(-1) - 2*2*2*2 -2*1*0*(-1) - 2*1*0*4)

det(A) = (0 + 4 + 0 + 0 + 0 + 0 + 12 + (-3) + 0 + 0 + 0 + 9 -(-4) - 2 - 2 - 0 - 0 - 0 - 0 - 24 - 0 - 0 - 16 - 0 - 0)

det(A) = (-35)

2. det(a) = 0 e det(a) != 0

A = [1 0 -1 2]
    [0 4 -2 0]
    [-1 0 1 2]
    [2 0 -2 4]
    
det(A) = 0

A = [1 0 2 0]
    [2 1 1 1]
    [2 3 0 1]
    [-1 1 2 2]
    
det(A) != 0  -> det(A) = 19

\end{lstlisting}

3. O código que replica a fórmula de Leibniz em python:

\begin{lstlisting}
    matriz = [[1, 0, -1, 2],
          [0, 4, -2, 0],
          [-1, 0, 1, 2],
          [2, 0, -2, 4]]

def leibniz(matriz):
    n = len(matriz)
    if n == 1:
        return matriz[0][0]
    else:
        soma = 0
        for j in range(n):
            nova_matriz = []
            for i in range(1, n):
                linha = []
                for k in range(n):
                    if k != j:
                        linha.append(matriz[i][k])
                nova_matriz.append(linha)
            sinal = (-1) ** j
            soma += matriz[0][j] * sinal * leibniz(nova_matriz)
        return soma

determinante = leibniz(matriz)

print("O determinante da matriz:", determinante)
\end{lstlisting}

\begin{figure}
    \centering
    \includegraphics{fotocerta.png}
    \label{fig:my_label}
\end{figure}

\end{document}
